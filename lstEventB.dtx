% \iffalse meta-comment
% 
% lstEventB.ins
% 
% Copyright (C) 2017 by Thai Son Hoang and Chenyang Zhu
% <T dot S dot Hoang and C dot Zhu at ecs dot soton dot ac dot uk>
% --------------------------------------------------------------------
% 
% This file may be distributed and/or modified under the
% conditions of the LaTeX Project Public License, either version 1.3
% of this license or (at your option) any later version.
% The latest version of this license is in:
% 
%      http://www.latex-project.org/lppl.txt
% 
% and version 1.3 or later is part of all distributions of LaTeX 
% version 2005/12/01 or later.
% 
% This work has the LPPL maintenance status "author-maintained".
% 
% The Current Maintainer of this work is T.S. Hoang and C. Zhu.
%
% This work consists of the files lstEventB.dtx, lstEventB.ins,
% the derived file lstEventB.sty, the generated documentation
% lstEventB.pdf, and some sample requirements documents.
% 
% \fi
% 
% \iffalse
%<lstEventB>\NeedsTeXFormat{LaTeX2e}\relax
%<lstEventB>\ProvidesPackage{lstEventB}
%<lstEventB>    [2017/06/06 v0.1 Package for listing Event-B code] 
% 
%<*driver> 
\documentclass[a4paper]{ltxdoc}
\usepackage{lstEventB}
\EnableCrossrefs
% ^^A\CodelineIndex
\PageIndex
\RecordChanges

\begin{document}
\DocInput{lstEventB.dtx}
\end{document}
%</driver>
% \fi
% 
% \CheckSum{0}
% 
% \CharacterTable
% {Upper-case    \A\B\C\D\E\F\G\H\I\J\K\L\M\N\O\P\Q\R\S\T\U\V\W\X\Y\Z
% Lower-case    \a\b\c\d\e\f\g\h\i\j\k\l\m\n\o\p\q\r\s\t\u\v\w\x\y\z
% Digits        \0\1\2\3\4\5\6\7\8\9
% Exclamation   \!     Double quote  \"     Hash (number) \#
% Dollar        \$     Percent       \%     Ampersand     \&
% Acute accent  \'     Left paren    \(     Right paren   \)
% Asterisk      \*     Plus          \+     Comma         \,
% Minus         \-     Point         \.     Solidus       \/
% Colon         \:     Semicolon     \;     Less than     \<
% Equals        \=     Greater than  \>     Question mark \?
% Commercial at \@     Left bracket  \[     Backslash     \\
% Right bracket \]     Circumflex    \^     Underscore    \_
% Grave accent  \`     Left brace    \{     Vertical bar  \|
% Right brace   \}     Tilde         \~}
% 
% 
% \changes{v0.0}{2016/06/06}{Initial version}
% 
% \GetFileInfo{lstEventB.sty}
% 
% \DoNotIndex{\\}
% \DoNotIndex{\DeclareOption}
% \DoNotIndex{\ProcessOptions}
% \DoNotIndex{\RequirePackage}
% \DoNotIndex{\arabic}
% \DoNotIndex{\begin}
% \DoNotIndex{\csname,\csuse}
% \DoNotIndex{\def,\do,\dolistloop}
% \DoNotIndex{\end,\endcsname,\expandafter}
% \DoNotIndex{\hline}
% \DoNotIndex{\ifstrequal,\iftoggle,\item}
% \DoNotIndex{\label,\labelformat,\listadd}
% \DoNotIndex{\medskip}
% \DoNotIndex{\newcommand,\newcounter,\newenvironment,\newtoggle,\nomenclature}
% \DoNotIndex{\quad}
% \DoNotIndex{\renewcommand,\renewenvironment,\ref,\refstepcounter}
% \DoNotIndex{\setcounter,\small}
% \DoNotIndex{\textsf,\textwidth,\togglefalse,\toggletrue}
% \DoNotIndex{\value}
% \DoNotIndex{\xspace}
%
% \title{The \textsf{lstEventB} package\thanks{This document
% corresponds to \textsf{lstEventB}~\fileversion, dated~\filedate.}}
% \author{Thai Son Hoang \\ ECS, University of Southampton \\ \texttt{<T dot S dot Hoang at ecs dot
% soton dot ac dot uk>}}
% \date{June 6, 2017}
% 
% \maketitle
% 
% ^^A %%%%% Abstract %%%%%
% \begin{abstract}
%   This package provides macros for typesetting Event-B code.  It
%   was developed at the University of Southampton.
% \end{abstract}
% 
% ^^A %%%%% Table of contents %%%%%
% \tableofcontents
% 
% ^^A %%%%% Introduction %%%%%
% \section{Introduction}
% 
% This package was developed in order to ease the listing of
% Event-B code in \LaTeX{}.
% 
% ^^A %%%%% Usage %%%%%%
% \section{Usage}
% 
% Just like any other package, you need to request this package with a
% |\usepackage| command in the preamble.
%
% So in the simpler case, one just types
% 
% \indent |\usepackage{lstEventB}|
%
% \noindent to load the package  
% 
% \StopEventually{
% \PrintChanges
% \PrintIndex
% }
%   
% ^^A %%%%% Implementation %%%%%
% \section{Implementation}
%
% ^^A %%% Package loading %%% 
% The implementation is quite straightforward.  Our implementation is
% based on the |listings| package.
% 
% \iffalse ^^A BEGIN Produce comments only in the resulting style file
%<lstEventB>
%<lstEventB>%%%%% BEGIN Package loading %%%%%
% \fi ^^A END Produce comments only in the resulting style file
%
%    \begin{macrocode}
\RequirePackage{listings}
\RequirePackage{xspace}
\RequirePackage{nomencl}
%    \end{macrocode}
%
% \iffalse ^^A BEGIN Produce comments only in the resulting style file
%<lstEventB>%%%%% END Package loading %%%%%
%<lstEventB>
% \fi ^^A END Produce comments only in the resulting style file
%
% \begin{macro}{\newabbrev}
%   \changes{v1.0}{2013/04/22}{Macro created}
%   The |newabbrev| makes use of the worker macro |newfullabbrev| for
%   creating abbreviations macros.
% 
% \iffalse^^A BEGIN Produce comments only in the resulting style file
%<*lstEventB>

% Meta-macro to create abbreviation macros.
%
% Arguments:
% 1. (Optional) String to be used as macro
% 2. The abbreviation (also used as the macro if the optional argument
%    is empty)
% 3. The full string expansion.
%
% Usage:
% - \newabbrev{SME}{Small and Medium-sized Enterprise} will create
% macros \SME will expand as Small and Medium-sized Enterprise (SME)
% and \SMEs will expand as Small and Medium-sized Enterprises (SMEs).
% - \newabbrev[randd]{R\&D}{Research and Development} will create
% macros \randd will expand as Research and Development (R\&D)
% and \randds will expand as Research and Developments (R\&Ds).
%</lstEventB>
% \fi^^A END Produce comments only in the resulting style file
%    \begin{macrocode}
\newcommand{\B@font}{\sffamily}
\newcommand{\setBfont}[1]{%
  \renewcommand{\B@font}{#1}%
}

\lstMakeShortInline[language=EventB, breaklines=f, basicstyle=\B@font\normal]|
\lst@definelanguage{EventB}{
  basicstyle=\B@font\scriptsize,
  keywords={
    context,sets,constants,axioms, do, od,%
    machine,sees,refines,variables,invariants,events,with,extended,theorem,%
    begin,then,end,when,any,where,theory,types,operators,for,prefix,infix,imports,
    associative,commutative
  },
  keywordstyle=\color{red}\bfseries,
  identifierstyle=\color{orange},
  sensitive=false,
  comment=[l]{//},
  morecomment=[s]{/*}{*/},
  commentstyle=\color{purple}\ttfamily,
  stringstyle=\color{blue}\ttfamily,
  morestring=[b]',
  morestring=[b]"
}

\lstset{
  columns=fullflexible,
  numberbychapter=false,
  frame=top,frame=bottom,
  basicstyle=\ttfamily\scriptsize,    % the size of the fonts that are used for the code
  stepnumber=1,                           % the step between two line-numbers. If it is 1 each line will be numbered
  numberstyle=\tiny,
  numbersep=5pt,                         % how far the line-numbers are from the code
  tabsize=2,                              % tab size in blank spaces
  extendedchars=true,                     %
  breaklines=true,                        % sets automatic line breaking
  captionpos=b,                           % sets the caption-position to top
  mathescape=false,
  stringstyle=\color{black}\ttfamily, % Farbe der String
  keywordstyle=\color{black}\ttfamily, % Farbe der String
  showspaces=false,           % Leerzeichen anzeigen ?
  showtabs=false,             % Tabs anzeigen ?
  xleftmargin=10pt,
  framexleftmargin=10pt,
  framexrightmargin=0pt,
  framexbottommargin=5pt,
  framextopmargin=5pt,
  showstringspaces=false,      % Leerzeichen in Strings anzeigen ?
  escapechar=\%,
  numbers=left,                    % where to put the line-numbers; possible values are (none, left, right)
  numbersep=5pt,
  inputencoding=utf8,         % UTF-8 input encoding
  extendedchars=true,         % Use extended characters
  literate=
  {⩥}{{$\ransub$}}1
  {▷}{{$\ranres$}}1
  {∩}{{$\binter$ }}1
  {∪}{{$\bunion$ }}1
  {↦}{{$\mapsto$ }}1
  {→}{{$\tfun$ }}1
  {⊈}{{$\nsubset$}}1
  {⊄}{{$\nsubseteq$}}1
  {∉}{{$\notin$ }}1
  {⇔}{{$\leqv$}}1
  {⇒}{{$\limp$}}1
  {∧}{{$\land$}}1
  {∀}{{$\forall$ }}1
  {}{{$\ovl$ }}1 
  {≠}{{$\neq$ }}1 
  {≤}{{$\leq$ }}1 
  {≥}{{$\geq$ }}1 
  {⊂}{{$\subset$ }}1 
  {⊆}{{$\subseteq$ }}1
  {↔}{{$\rel$ }}1
  {⤖}{{$\tbij$ }}1
  {⇸}{{$\pfun$ }}1
  {⤔}{{$\pinj$ }}1
  {↣}{{$\tinj$ }}1
  {⤀}{{$\psur$ }}1
  {↠}{{$\tsur$ }}1
  {∅}{{$\emptyset$ }}1
  {∖}{{$\setminus$ }}1
  {×}{{$\cprod$ }}1 
  {⊗}{{$\dprod$ }}1
  {∼}{{$\sim$}}1
  {⩤}{{$\domsub$ }}1
  {◁}{{$\domres$ }}1
  {λ}{{$\lambda$ }}1
  {‥}{{$\upto$ }}1
  {·}{{$\qdot$ }}1
  {^}{{$\hat$ }}1
  {≔}{{$\bcmeq$ }}1
  {:∈}{{$\bcmin$ }}1
  {:∣}{{$\bcmsuch$ }}1
  {∈}{{$\in$ }}1
  {↦}{{$\mapsto$ }}1
  {ℕ1}{{$\natn$ }}1 
  {ℕ}{{$\nat$ }}1
  {ℙ1}{{$\pown$ }}1 
  {ℙ}{{$\pow$ }}1 
  {ℤ}{{$\intg$ }}1 
  {⋂}{{$\Inter$ }}1 
  {⋃}{{$\Union$ }}1 
  {∨}{{$\lor$ }}1 
  {⊤}{{$\btrue$ }}1 
  {⊥}{{$\bfalse$ }}1
  {∘}{{$\circ$ }}1
  {⦂}{{$\oftype$ }}1  
  {∣}{{$\mid$ }}1 
  {+}{{$+$ }}1 
  {−}{{$-$ }}1 
  {*}{{$*$ }}1 
  {÷}{{$/$ }}1 
  {ℝ}{{$\mathbb{R}$ }}1 
  {prj1}{{$\prjone$ }}1 
  {prj2}{{$\prjtwo$ }}1 
  {==}{{$\widehat{=}$ }}1 
  {¬}{{$\neg$ }}1 
, % Special symbols
 }

%    \end{macrocode}
% \end{macro} ^^A \newabbrev
%
%
% \Finale
\endinput